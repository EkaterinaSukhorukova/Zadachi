\documentclass[a4paper,10pt]{article}
\usepackage[utf8]{inputenc}
\usepackage[english, russian]{babel}


\title{Исключения}

\begin{document}
\maketitle

Исключительная ситуация (исключение) - ошибка, возникающая во время выполнения программы. При их возникновении во время работы программы автоматически вызывается обработчик исключений.

Управление механизмом обработки исключений держится на трех ключевых словах: try, catch и throw. Программные инструкции, которые вы считаете нужным проконтролировать на предмет исключений, помещаются в try-блок. Если там возникает исключение, оно дает о себе знать выбросом информации, благодаря throw. Эта информация перехватывается catch-блоком, который обрабатывает данное исключение.
\\
\\Пример: 
\\
\hline
int main ()
\\{
\\	int a,b,c;
\\	cin>>a>>b;
\\	try
\\	{
\\		if (b==0) throw 99;
\\		else c=a/b;
\\	}
\\	catch (int i)
\\	{
\\		cout<<"На ноль делить нельзя!!!";
\\}
\\	cout<<"Конец";
\\	return 0;
\\}
\hline
\\

* Если в try-блоке исключение не генерировалось, catch-блок не сработает. Программа его обойдет

* Catch-инструкций может быть несколько, каждая перехватывает свой тип исключения (в примере это int)

* exit (EXIT-SUCCESS) - завершение работы программы, с сообщением об успешном окончании; exit (EXIT-FAILURE) - завершение работы программы, с сообщение о неудачном окончании; abort () - просто завершение работы программы. Чтобы пользоваться этими функциями необходимо подключить <cstdlib>

* Catch-выражение для базового класса перехватит исключение любого производного типа. Поэтому, если необходимо перехватывать исключения как базового, так и производного типа, в сatch-последовательности catch-инструкцию для производного типа необходимо поставить перед catch-инструкцией для базового типа.

Существует catch-инструкция перехватывающая исключения всех типов:
\\\fcolorbox{black}{white}{catch (...) {Тело catch-инструкции};} //... - в прямом смысле.
\\

* Обычно перехват всех исключений помещают последним в очереди.
\\

Существуют средства, которые позволяют ограничить тип исключений, которые может генерировать функция за пределами своего тел. Для этого нужно внести в определение функции throw-выражения:
\\"тип возвращаемого значения" "имя функции" (аргументы) throw (список имен типов) {Тело функции}
Пример:
\\\fcolorbox{black}{white}{void Xhandler (int test) throw (int, char, double)}
\\

* Можно повторно сгенерировать исключение в его обработчике (см. повторное генерирование исключения)
\\

\\Пример по исключениям, когда ошибка ловится, но данные все равно потеряны:

\\- Работаем с текстом, в котором записаны числа. Мы переводим числа из типа чар в инт и пытаемся выполнить над ними какие-то арифметические действия, все в том де тексте. 

\\Ошибка отлавливается, переведенные числа записываются в заранее определенные переменные, но, к примеру, не учитываются пробелы или знаки препинания (точка в конце строки) и действие все равно происходит не так, как мы хотим.
\\
\\Представим, что мы работаем с некоторым классом Myclass и у нас есть переменные А и В, на классе определен оператор =. Представим такой код:
\\
\hline
\\
\\Try {
\\Myclass T = A;
\\A=B;
\\B=T;
\\}
\hline
\\

Если ошибка произойдет в последней строчке, то у нас теряется значение А, а в В лежит то же значение.
\\В таких случаях нужно заранее думать о том, что ошибка может возникнуть. И нужно заранее готовиться к возможности возникновения этой ошибки.
\\
\\Что важно запомнить об исключениях:
\begin{itemize}
\item try-блок — так называемый блок повторных попыток. В нем надо располагать код, который может привести к ошибке и аварийному закрытию программы;
\item throw генерирует исключение. То что остановит работу try-блока и приведет к выполнению кода \item catch-блока. Тип исключения должен соответствовать, типу принимаемого аргумента catch-блока;
\item catch-блок — улавливающий блок, поймает то, что определил  throw и выполнит свой код. Этот блок должен располагаться непосредственно под try-блоком. Никакой код не должен их разделять.
если в try-блоке исключение не генерировалось, catch-блок не сработает. Программа его обойдет.
\end{itemize}

\end{document}
