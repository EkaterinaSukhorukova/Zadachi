\documentclass[a4paper,10pt]{article}
\usepackage[utf8]{inputenc}
\usepackage{color}
\usepackage[english, russian]{babel}


\title{Преобразование типов}

\begin{document}
\maketitle

\\int a=5, b=8;
\\double c=b/a
\\c=1.0
\\Избежать можно заранее приведя типы (в скобочках) или умножив на 1.0.

\\С операторами есть много незаконных преобразований типов. 
\\

Пример:
\\
\hline
\\char a=‘1’;

\\int c = *((int*) & a)
\\
\hline
\\

\\В этом случае мы берем не одну ячейку, а 4, потому что инт занимает 4 бита. Но мы не можем быть уверены, что у нас есть доступ в эти ячейки. 
\\

\hline
\\Class A {

\\double b;

\\char c;}
\\
\hline
\\

\\Нам нужно: обратиться и изменить объект класса не из дружественной функции, а откуда-нибудь. (A*Q или A Q)

\\У нас есть указатель на начало объекта. Если мы сделаем указатель на начало объекта, присвоим ему значение int, изменим это значение и разименуем, то мы изменим значение:
\\

\\а (*((int*) Q) = 2).

\\Чтобы перейти к значению double:

\\*((double*) Q) + 1) =2.0

\\Есть знак #pragma, который позволяет управлять как временем, так и памятью.

\\Чтобы перейти к значению чар:

\\*(((char*) Q) + 16) = ‘2’ – самое простое.

\\*((char*) ((double*) Q +2)) = ‘2’.
\\
\\
\\

\\О приведении типов уже говорилось ранее в разделе наследования. 
\\

\hline

\\Class A {

\\virtual f / {…}

\\}
\\

\\Class B: A {

\\f () {

\\…

\\}

\\}
\\

\\B temp

\\A* ptr-temp = & temp

\\ptr-temp -> f ();
\\
\hline
\\


\\Можем не задумываться, для каких типов мы вызываем функцию f. 
\\

\hline

\\Int a=5;

\\long long =a;

\\f (int a)

\\f (long a)
\\


\\long a = 42;

\\f(a)
\\
\hline
\\
\\

\\Вызовется функция от int, т.е. функция отработает не так, как мы ожидаем. 
\\

\\Вырожденное преобразование типов (касты)
\\
\begin{itemize}
    \item static\underline{ }cast. 
    \\Преобразование между связанными значениями.
    \\Связанность проверяется на этапе компиляции, поэтому и называется static.
    \\Типы, к котрым применим static\underline{ }cast:
    \\ - числовые
    \\\fcolorbox{black}{white}{static\underline{ }cast<int>(5.5);}
    \\ - типы, связанные наследованием
    \\\fcolorbox{black}{white}{B * b = ...; 
        \\ A * a = static\underline{ }cast<A *>(b); 
        \\ b = static\underline{ }cast<B *>(a);}
    \\ - пользовательские преобразования
    \\\fcolorbox{black}{white}{static\underline{ }cast<string>("Hello");}
    \\ - к void *
    \\ \fcolorbox{black}{white}{ struct A; struct B; struct D; A*a=...; B*b=static\underline{ }cast<B*>(a); B* b=(B*)a;}
    
    \item dynamic\underline{ }cast/
    \\Осуществляет безопасное преобразование указатели или ссылки на базовый класс в указатель или сссылку на произвольный класс.
    \\Используется для:
    \\ - Приведения типов
    \\\fcolorbox{black}{white}{Circle*c=dynamic\underline{ }cast<Circle>(a)}
    \\ - Преобразования указателей
    \\\fcolorbox{black}{white}{A* a=newC(); B*b=dynamic\underline{ }cast<B*>(a);}
    \\ - Преобразованиея ссылок
    \\\fcolorbox{black}{white}{B&b=dynamic\underline{ }cast<B&>(*a);}
    \item reinterpret\underline{ }cast
    \\Приводит друг к другу указатели, которые друг от друга не зависят, не меняя константности
\\
Оператор reinterpret\underline{ }cast является жестко машинно-зависимым. Чтобы безопасно ипользовать оператор reinterpret\underline{ }cast, следует хорошо понимать, как именно реализованы используемые типы, а также то, как компилятор осуществляет приведение
\\ 

Пример:
\\\fcolorbox{black}{white}{int* p=...; double*d=reinterpret\underline{ }cast<int*>(p);}
\\
\item const\underline{ }cast
\\Позволяет убрать или добавить константность.
\\\fcolorbox{black}{white}{A const* ac=...; A* a=const\underline{ }cast<A*>(ac);}
\\
\\const\underline{ }cast не применим:
\\ - Для тех объектов, которые объявлены как const, нельзы отменять константность. Будет undefined behaviour.
\\

\\ - Если в функцию, например, передана константная ссылка, то внутри мы можем снять константность, хотя это не хорошо.
\\

\\ - Для приведения констант применяется лишь оператор const\underline{ }cast. Применение любого другого оператора приведения типов в данном случае привело бы к ошибке при компиляции. Аналогично, ошибка при компиляции произойдет в случае использования оператора const\underline{ }cast в записи, которая осуществляет любые другие преобразования типов, отличные от создания или удаления константности


\end{itemize}
\\

\\Все это – шаблонизируемые функции.
\\

\\Статическое преобразование:
\\int a = 1000;
\\Char c = static-cast <char> (a);
\\
\\

\\Бонусы:
\\1.	Контроль. Мы сами управляем тем, что и во что преобразовывается.
\\2.	Компилятор сам проверяет, что у нас было получено
f (static-cast <long> (146))
\\3.	Когда код большой это можно использовать для самоконтроля и поддержки самого кода. 
\\
\\


\\Динамическая типизация:
\\Приведение и корректность проверяется и происходит во время исполнения, в то время как у статического преобразования все это происходит во время компиляции.
\\

\\Пример:
\\Пусть есть объект, конструктор для которого не отработал. Например наследование виртуальной функции (смотри ранее)
\\

\\void foo (A& a) {
\\try {
\\B & b = dynamic_cast <&b> (a) (объект а определен ранее)
\\b.f();
\\}

\\Catch {const std: bad_cast& e}
\\

\\}

\\Приведение ссылок проверит, объектом какого класса является объект а. 
\\

\\Class A {
\\foo ();
\\}
\\

\\Class B: A {
\\bar ();
\\}
\\

\\В таком случае приведение выкинет исключение конкретного вида, которое мы можем поймать:
\\Catch {const std: bad_cast& e}
\\

\\В принципе, без кастов всегда можно обойтись. Эта некоторая заплатка, которая помогает оптимизировать код, и ее всегда видно, потому что она довольно громоздкая.
\\
\\

\\Интерпретированное преобразование.
\\reinterpret
\\Каламбур типизации. Когда мы делаем совершенно незаконное преобразование типов. 

\end{document}
